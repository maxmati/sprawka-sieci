\chapter{Lab 56: HP ProCurve}

\section{Konfiguracja}

\subsection{Ogólne}
\cmdc{config}{telnet-server}{Włączenie serwera telnet}
\cmdc{config}{no telnet}{Wyłączenie serwera telnet}

\cmdc{config}{web-management}{Włączenie serwera WWW}

\cmdc{config}{ip ssh key-size \textit{rozmiar\_klucza} }{Ustawienie rozmiaru klucza SSH}
\cmdc{config}{crypto key generate}{Generowanie klucza SSH (async)}
\cmdc{config}{ip ssh port \textit{numer\_portu} }{Ustawienie portu na którym ma działać serwer SSH}
\cmdc{config}{ip ssh}{Włączenie serwera SSH}

\cmdc{config}{interface \textit{numer}}{Przejście do konfiguracji portu}
\cmdc{config}{ip route \textit{netowork} \textit{mask} reject}{Zdefiniowanie reguły null}

\subsection{VLAN}
W przypadku przełączników HP w przeciwieństwie do tych Cisco przypisujemy porty do vlan'ów a nie vlan'y do portów

\cmdc{vlan-\textit{numer}}{tagged \textit{numer\_lub\_zakres\_portów}}{Przypisanie portu do Vlana}

\section{Diagnostyka}

\subsection{Ogólne}
\cmde{show telnet}{Wyświetlenie informacji o stanie serwera telnet}
\cmde{show ip ssh}{Wyświetlenie informacji o stanie serwera SSH}

\subsection{VLAN}
\cmde{show vlan}{Wyświetlenie informacji o stanie bazy vlan}
\cmde{show vlan \textit{numer}}{Wyświetlenie informacji konkretnym vlan'ie}