\documentclass[]{report}

%\usepackage{polyglossia}
\usepackage[utf8]{inputenc}
\usepackage{polski}
\usepackage{hyperref}
\hypersetup{
	colorlinks,
	citecolor=black,
	filecolor=black,
	linkcolor=black,
	urlcolor=black
}
%\usepackage[toc,page]{appendix}

\usepackage[scale=0.8,a4paper]{geometry}

\newcommand{\cmdc}[3]{\paragraph{ \texttt{(#1)\# #2} \newline #3 }}
\newcommand{\cmde}[2]{\paragraph{ \texttt{\# #1} \newline #2 }}

% Title Page
\title{Sieci Kompuerowe -- sprawozdanie z ćwiczeń laboratoryjnych}
\author{Mateusz Nowotyński oraz Marcin Moskal\\
	Zespół nr. 7\\
	wtorek godzina: 8:00}

\begin{document}
\maketitle

\tableofcontents

% Switch I
\chapter{Lab 21: Cisco Catalyst -- VLAN, VTP}

\section{Konfiguracja}

\cmdc{config}{no ip domain-lookup}{wyłączenie rozwiązywania nazw DNS}\subsection{Ogólne}
\cmdc{config-if}{swichport nonegotiate}{Wyłączenie autonegocjacji}
\cmdc{config}{interface range \textit{nazwa\_interfejsu}/\textit{gniazdo\_pierwsze} - \textit{gniazdo\_ostatnie}}{Wejście w tryb konfiguracji wielu portów jednocześnie (zakresu)}

\subsection{VLAN}
Domyślnie port znajduje się w trybie \texttt{dynamic} czyli adaptuje się do portu po drugiej stronie. Baza danych VLAN trzymana jest w pliku \texttt{vlan.dat} (nie w \texttt{startup-config}!).

\subsubsection{Konfiguracja}
\cmdc{config-if}{switchport mode \{access,trunk,dynamic\}}{zmiana trybu pracy przełącznika (dynamic - uzgadniany z drugą stroną)}
\cmdc{config}{vlan \textit{numer\_vlan}}{tworzenie nowego vlanu}
\cmdc{config-if}{switchport access vlan \textit{numer\_vlan}}{przypisanie pojedynczego portu do vlan}
\cmdc{config-if}{switchport trunk encapsulation dot1q}{ustawienie trybu enkapsulacji vlanów na 802.1q}
\cmdc{config-if}{switchport trunk native vlan \textit{numer\_vlan}}{ustawianie natywnego VLAN}
\cmdc{config-if}{switchport trunk allowed vlan \textit{pierwszy\_vlan}-\textit{ostatni\_vlan}}{Zezwolenie na przesyłanie wybranych vlanów przez port trunk}
\cmdc{config-if}{switchport trunk allowed vlan remove 10}{Usunięcie wybranego VLAN z trunku}

\subsubsection{Diagnostyka}
\cmde{show interfaces trunk}{Wyświetla enkapsulację dla interfejsów trunk oraz numery VLAN dopuszczonych do ruchu}
\cmde{show interfaces \textit{nazwa\_interfejsu} switchport}{Wyświetla informacje o trybie pracy interfejsu (trunk/access, voice VLAN, pruning)}
\cmde{show interface [\textit{nazwa\_interfejsu}] status}{Wyświetla listę (interfejs) ze statusem, numerem VLAN (ewentualnie trunk), duplex, prędkość i typ}
\cmde{vlan database}{Wchodzi w tryb konfiguracji bazy danych VLAN}

\subsection{VTP (\textit{Virtual Trunking Protocol})}
VTP pracuje w trzech trybach:
\begin{itemize}
	\item client -- konfiguracja pobierana jest z serwera oraz przekazywana dalej w sieć
    \item server -- utworzone na tym urządzeniu VLAN-y są propagowane do switchów pracujących w trybie VTP client
    \item transparent -- konfiguracja jest przesyłana dalej, bez interpretacji na urządzeniu
\end{itemize}

\subsubsection{Konfiguracja}
\cmdc{config}{vtp domain \textit{nazwa\_domeny}}{Ustawienie domeny VTP (musi byc zgodna na wszytkich switchach)}
\cmdc{config}{vtp password \textit{hasło}}{Ustawienie hasła na VTP (wymagana zgodność na wszystkich switchach)}
\cmdc{config}{vtp mode \{client,server,transparent\}}{Ustawienie trybu pracy VTP}

\subsubsection{Diagnostyka}
\cmde{show vtp status}{Wyświetlenie informacji o stanie VTP}
\cmde{debug sw-vlan vtp events}{Uruchomienie debugowania VTP}

% Switch II
\chapter{Lab 22: Cisco Catalyst -- \texttt{telnet}, SSH, STP, EtherChannel}

\section{Konfiguracja dostępu zdalnego}
Najczęściej w urządzeniach znajduje się 16 (0 -- 15) terminali wirtualnych VTY. \newline Dostęp do trybu \texttt{enable} zdalnie jest możliwe jedynie po uprzednim zabezpieczeniu wejścia do trybu hasłem.
\cmdc{config}{line vty \textit{pierwsze\_vty} \textit{ostatnie\_vty}}{Wejście w tryb konfiguracji VTY}

\subsection{Konfiguracja \texttt{telnet}}
Usługę uruchamiamy na konkretnych liniach VTY (wejście do trybu \texttt{config-line} i konfiguracja opisane są powyżej). \newline
\texttt{telnet} pracuje na porcie 23 TCP.

\cmdc{config-line}{password \textit{hasło}}{Ustawienie lokalnego hasła dla wybranych linii VTY}
\cmdc{config-line}{login}{Ustanowienie nakazu używania lokalnego hasła przy logowaniu}
(nie działa? -- patrz: wyłączanie autoryzacji new-model)
\cmdc{config}{no aaa new-model}{Wyłączenie trybu autoryzacji typu new-model}
\cmdc{config-line}{transport input telnet}{Zezwolenie na ruch \texttt{telnet} z wykorzystaniem wybranych linii}

\subsection{Konfiguracja SSH}
Usługę uruchamiamy na konkretnych liniach VTY (wejście do trybu \texttt{config-line} i konfiguracja opisane są powyżej) za pomocą ostatniego polecenia. \newline
Konieczne jest wcześniejsze ustawienie nazwy domeny, wygenerowanie klucza oraz zdefiniowanie użytkownika (polecenia poniżej).

\cmdc{config}{ip domain-name \textit{domena}}{Konfiguracja domeny}.
\cmdc{config}{crypto key generate rsa}{Generowanie klucza RSA o zadanej długości} (pytanie o długość klucza pojawia się interaktywnie)
\cmdc{config}{username \textit{nazwa\_użytkownika} priv \textit{priorytet\_użytkownika} password \textit{typ\_hasła} \textit{hasło}}{Definiowanie nowego użytkownika, gdzie: \textit{priorytet\_użytkownika} - wartość 15 jest najwyższa; \textit{typ\_hasła} - kod 0 oznacza hasło zadane jawnym tekstem, kod 7 - hasło szyfrowane}
\cmdc{config}{aaa new-model}{Pozwolenie na wykorzystanie lokalnego systemu kont (nie jest możliwe logowanie za pomocą hasła przypisanego do linii VTY)}
\cmdc{config-line}{transport input ssh}{Zezwolenie na ruch \texttt{telnet} z wykorzystaniem wybranych linii}

\subsubsection{Diagnostyka}
\cmde{show ip ssh}{Sprawdzenie konfiguracji SSH}

\subsubsection{Uwagi i problemy}
\begin{itemize}
	\item Logowanie przez SSH działa w trybie \texttt{aaa new-model}, zatem polecenie \texttt{login} przestaje działać.
	\item Ustawienie długości klucza poniżej 512 bitów może powodować problemy podczas logowania (PuTTY odmawia logowania z powodu zbyt krótkiego klucza).
\end{itemize}

\subsection{Zarządzanie dostępem do linii VTY za pomocą ACL}
Dostęp do linii może być zarządzany poprzez przypisanie listy ACL (standardowej, rozszerzonej, nazwanej) -- patrz laboratorium ACL.

\cmdc{config-line}{access-class \textit{numer\_listy\_ACL} in}{Przypisanie listy do linii VTY}

\cmdc{config}{access-list \textit{numer\_listy\_standardowej} permit \textit{adres} \textit{inwersja\_maski}}{Tworzenie przykładowej standardowej listy ACL} (do konfiguracji można wykorzystywać także inne rodzaje list)

\section{STP -- \textit{Spanning Tree Protocol}}

\subsection{Założenia}
\subsubsection{Zasada działania}
\begin{itemize}
	\item W segmencie sieci istnieje \textit{jeden} wyróżniony przełącznik, będący korzeniem drzewa rozpinającego (tzw. \textit{root bridge}),
	\item Przełączniki nie będące root bridge układają scieżkę do korzenia wycinając wszystkie redundantne ścieżki stanowiące pętle w sieci.
\end{itemize}
\subsubsection{Nomenklatura}
\begin{itemize}
	\item root bridge -- korzeń drzewa rozpinającego,
	\item Root ID -- identyfikator przełącznika, będącego korzeniem,
	\item Bridge ID -- identyfikator bieżącego przełącznika,
	\item root port -- port, który z bieżącego przełącznika prowadzi do korzenia -- w górę drzewa (nie występuje w korzeniu),
	\item designated ports -- porty odpowiedzialne za przekazywanie ruchu w dół drzewa (TODO: zrobić rozeznanie definicji),
	\item alternative ports -- (dla PVST i RSTP) -- porty prowadzące do korzenia, nie będące najlepszym wariantem,
	\item BPDU -- \textit{Bridge Protocol Data Unit} -- rozgłaszane ramki STP.
\end{itemize}

\subsection{Konfiguracja}
\subsubsection{Wymuszanie roli}
\cmdc{config}{spanning-tree vlan \textit{numer\_VLAN} root primary}{Przeniesienie roli \textit{root bridge} na bieżące urządzenie (manipulacja dokonywana jest przez zmniejszenie priorytetu o 8192)}
\cmdc{config}{spanning-tree vlan \textit{numer\_VLAN} root secondary}{określa \textit{priority} jako wartość ,,drugą najlepszą'' -- powodując przejęcie funkcji \textit{root bridge} gdy obecny korzeń ulegnie awarii (poprzez zmniejszenie priority o 4096)}

\subsubsection{Ręczna konfiguracja parametrów}
Kolejność wyboru \textit{root port}: 1. koszt (niższy wygrywa), 2. przysłany priorytet (niższy wygrywa), 3. numer portu (młodsza część priorytetu)
\cmdc{config}{spanning-tree vlan \textit{numer\_VLAN} priority \textit{priorytet}}{Ręczna konfiguracja priorytetu BID (\textit{Bridge ID}) BPDU, gdzie priorytet jest wielokrotnością 4096}
\cmdc{config-if}{spanning-tree port-priority \textit{priorytet}}{Ręczna konfiguracja priorytetu portu (granulacja 16, wartość domyślna 128)}
\cmdc{config-if}{spanning-tree port-priority \textit{priorytet}}{Ręczna konfiguracja priorytetu portu dla VLAN -- jeden port, osobne priorytety dla każdego VLAN}
\cmdc{config-if}{spanning-tree cost \textit{koszt}}{Ręczna konfiguracja kosztu (10Mbps = 100, 
	100Mbps = 19, 1Gbps = 4, 10Gbps = 2)}

\subsubsection{Wybór wariantu STP}
\cmdc{config}{spanning-tree mode pvst}{Włączenie PVST (\textit{per-VLAN} STP)}
\cmdc{config}{spanning-tree mode rapid-pvst}{Włączenie Rapid-PVST (\textit{rapid per-VLAN STP)}}

\subsubsection{Wyłączenie STP}
\cmdc{config}{spanning-tree portfast default}{Wyłączenie funkcjonowania STP dla całego przełącznika}
\cmdc{config-if}{spanning-tree portfast}{Wyłączenie funkcjonowania STP dla pojedynczego portu}

\subsection{Zabezpieczenia \texttt{bpduguard} i \texttt{guard root}}
\cmdc{config-if}{spanning-tree bpduguard disable}{Wyłączenie blokady BPDU -- wyłączenie ignorowania BPDU spreparowanych przez stacje końcowe}
\cmdc{config-if}{spanning-tree guard root}{Zablokowanie statusu \textit{root bridge} -- blokada portu przy próbie przejęcia roli korzenia przez inny przełącznik}

\subsection{Diagnostyka}
\cmde{show spanning-tree}{TODO: sprawdzić, co wyświetla}
\cmde{show spanning-tree summary}{TODO: sprawdzić, czym się różni od powyższego}
\cmde{show spanning-tree detail}{TODO: sprawdzić, czym się różni od powyższego}
\cmde{show spanning-tree vlan \textit{numer\_VLAN} [detail])}{Wyświetlanie stanu STP dla wybranych VLAN}
\cmde{show spanning-tree vlan \textit{pierwszy\_VLAN}-\textit{ostatni\_VLAN}}{Wyświetlanie stanu STP dla wybranych VLAN}
\cmde{show spanning-tree vlan \textit{numer\_VLAN} interface \textit{numer\_interfejsu}) [detail]}{Wyświetlanie stanu STP dla wybranego interfejsu (TODO: sprawdzić, gdzie występuje parametr detail)}

\subsubsection{Debugowanie}
\cmde{debug spanning-tree events}{Wyświetlanie komunikatów zdarzeń STP}

\section{Zarządzanie adresami MAC i filtrowanie}
Obowiązuje notacja adresów ,,\texttt{aabb.ccdd.eeff}''.
\cmdc{config}{mac-address-table static \textit{adres\_MAC} \textit{numer\_VLAN} interface \textit{numer\_interfejsu}}{Dodanie statycznego wpisu do tablic MAC}
\cmdc{config}{mac-address-table static \textit{adres\_MAC} vlan \textit{numer\_VLAN} drop}{Odrzucanie ruchu zawierającego podany adres MAC (notacja aabb.ccdd.eeff)}


\subsection{Diagnostyka}
\cmde{show mac-address-table}{Wyświetlenie tablicy znanych adresów MAC}
\cmde{show mac address-table}{Wyświetlenie tablicy znanych adresów MAC (Cisco 2960)}
\cmde{show mac-address-table interface \textit{numer\_interfejsu}}{Wyświetlenie adresów MAC skojarzonych z interfejsem}

\section{EtherChannel}
Najpierw przeprowadzamy konfigurację, potem podpinamy okablowanie -- niespójna konfiguracja przeciwległych urządzeń spowoduje awaryjne wyłączenie portów (\texttt{err-disabled})!
Wszystkie porty zagregowane przez EtherChannel muszą pracować w tym samym trybie, tj. \texttt{access} albo \texttt{trunk}.

\subsection{Konfiguracja}
\cmdc{config}{interface Port-channel \textit{numer\_interfejsu}}{Tworzenie nowego interfejsu typu Port-Channel}
\cmdc{config-if[-range]}{channel-group \textit{numer\_EtherChannel} mode on}{Włączenie agregacji portów EtherChannel}

\subsection{Diagnostyka}
\cmde{show etherchannel summary}{Podsumowanie statusu EtherChannel}
\cmde{debug EtherChannel}{Debugowanie EtherChannel}

% Route I
%\include{lab70}
\chapter{Lab 37: Routery Cisco -- Serial, PPP (PAP, CHAP, Multilink)}
\section{Konfiguracja parametrów łącza}
\cmdc{config-if}{encapsulation \{PPP,HDLC\}}{Wybór enkapsulacji: PPP -- \textit{Point-to-Point Protocol}, HDLC -- \textit{High level Data Link Control}}
\cmdc{config-if}{clock rate \texttt{częstotliwość\_zegara}}{Wybór częstotliwości taktowania zegara łącza \underline{po stronie DCE}}
\subsection{Diagnostyka}
\cmde{show ip interface serial \texttt{numer\_interfejsu}}{Sprawdzenie stanu połączenia szeregowego}
\cmde{show controllers serial \texttt{numer\_interfejsu}}{TODO: sprawdzić, czym się różni od powyższego}
\cmde{debug ppp authentication \texttt{numer\_interfejsu}}{TODO: sprawdzić, czym się różni od powyższego}

\section{PPP}
\subsection{PAP -- autentykacja \texttt{plaintext}}
Nazwa użytkownika i hasło są dowolne. Bazą kont do autentykacji PAP jest lokalna baza routera. Poniższa konfiguracja demonstruje autentykację \textit{w jedną stronę} -- sprawdzana jest wiarygodność zdalnego routera przez lokalny.
\subsubsection{Konfiguracja lokalna -- serwera}
\cmdc{config}{username \textit{nazwa\_użytkownika} password \textit{hasło}}{Tworzenie konta dla użytkownika zdalnego}
\cmdc{config-if}{ppp authentication pap}{Włączenie autentykacji PPP PAP}

\subsubsection{Konfiguracja zdalna -- klienta}
\cmdc{config-if}{ppp pap sent-username \textit{nazwa\_użytkownika} password \textit{hasło}}{Konfiguracja danych logowania wysyłanych do serwera}

\subsection{CHAP -- autentykacja \texttt{challenge-response handshake}}
Zdalny router posiada lokalne konto o nazwie użytkownika takiej, jak nazwa hosta i wspólnym haśle (sprawdzana jest zgodność skrótu MD5). Polecenia są identyczne dla obu końców z wyjątkiem nazwy użytkownika -- wpisujemy \texttt{hostname} przeciwległego routera.
\cmdc{config}{username \textit{hostname\_odległego\_końca} password \textit{hasło}}{Tworzenie konta dla przeciwległego końca}
\cmdc{config-if}{ppp authentication chap}{Włączenie autentykacji PPP CHAP}

\subsection{Diagnostyka autentykacji}
\cmde{debug ppp authentication \texttt{numer\_interfejsu}}{Włączenie debugowania autentykacji PPP}

\subsection{\texttt{multilink} PPP -- agregacja portów}
Adresację IP prowadzimy na interfejsie \texttt{multilink}, a nie jego składowych.
\cmdc{config}{interface multilink \textit{numer\_interfejsu}}{Tworzenie nowego interfejsu typu \texttt{multilink}}
\cmdc{config-if}{encapsulation ppp}{Włączenie enkapsulacji PPP}
\cmdc{config-if}{ppp multilink group \textit{numer\_interfejsu\_multilink}}{Dołączenie interfejsu fizycznego do interfejsu \texttt{multilink}}
\cmdc{config-if}{ppp multilink fragment size \textit{wielkość\_fragmentu}}{Ustawienie wielkości fragmentu \underline{dla całego} interfejsu \texttt{multilink}}

\section{TODO: Łącza async AUX, HSSI, modem} % brakuje async AUX, HSSI, modem
\chapter{Lab 72: ACL}
Kolejne wpisy listy wprowadzamy w konwencji \textit{od szczegółu do ogółu}.
\section{Polecenia ogólne}
\cmdc{config-if}{ip access-group \textit{numer\_listy} \{in, out\}}{Przypisanie listy do interfejsu; \texttt{in} -- ruch przychodzący, \texttt{out} -- ruch wychodzący}
\cmdc{config}{ip access list resequence \textit{numer\_listy} \textit{nowy\_początek\_numeracji} \textit{przyrost}}{Przebudowa numeracji reguł danej listy ACL}
\cmdc{config}{access-list \textit{numer\_listy} remark \textit{komentarz}}{Dodaje komentarz do listy}

\section{Listy standardowe (1-99)}
Działają na adresach IP ze sprecyzowanymi maskami, nie rozróżniają źródła i celu.
\cmdc{config}{access-list \textit{numer\_listy} \{permit, deny\} \{any, host adres\_IP, adres\_IP maska\_odwrócona\}}{Dodanie wpisu do listy standardowej}
\cmdc{config}{ip access list standard \textit{numer\_listy}}{Wejście w tryb modyfikacji listy \texttt{config-std-nacl}}
\cmdc{config-std-nacl}{\textit{numer\_sekwencji} \{wpis\_ACL\}}{Modyfikacja pojedynczego wpisu o zadanym numerze sekwencji}
\cmdc{config-std-nacl}{no \textit{numer\_sekwencji}}{Kasowanie wpisu o zadanym numerze sekwencji}

\section{Listy rozszerzone (100-199)}
TODO: to jest do przepisania


Ilość kombinacji jest rozbudowana; zamiast gramatyki podane są przykładowe użycia:
\cmdc{config}{access list 190 deny tcp any 10.10.10.0 0.0 0.0.0.255 eq 23}{Zablokowanie transmisji TCP przychodzących z dowolnego źródła do sieci 10.10.10.10/24 z portem docelowym \{23, telnet\} (\texttt{telnet})}
\cmdc{config}{access list 190 deny tcp any eq \{23, telnet\} any eq 40}{Zablokowanie transmisji TCP *:23 $\leftrightarrow$ *:40}
\cmdc{config}{access list 190 deny tcp 10.1.1.1 0.0.0.1 eq 23 200.200.200.1 0.0.0.1 eq 100 }{Zablokowanie transmisji TCP: 10.1.1.0/31:23 $\leftrightarrow$ 200.200.200.0/31:100}
TODO: Sprawdzić powyższe
\cmdc{config}{access list 190 deny tcp host 10.10.10.1 any}{Zablokowanie transmisji TCP: 10.10.10.1:* $\leftrightarrow$ *:*}
\cmdc{config}{access list 190 deny \{ip, icmp\} any 10.10.10.0 0.0.0.255 }{Zablokowanie ruchu \{IP, ICMP\} * $\leftrightarrow$ 10.10.10.0/24}

\section{Listy nazwane}
Działają identycznie jak listy standardowe/rozszerzone, jednak odwołujemy się do nich przez nazwę, nie liczbę.
\cmdc{config}{ip access-list \{standard, extended\} \textit{nazwa\_listy}}{Tworzenie nowej listy nazwanej}

\section{Diagnostyka}
\cmde{show access-lists}{Wyświetlanie zapisanych list ACL}
\cmde{show access-list \textit{numer\_listy} [interface \textit{numer\_interfejsu} \{in, out\}]}{Wyświetlanie szczegółów ACL (ilość dopasowań listy ACL)}
\cmde{clear access-list counters}{Kasowanie liczników dopasowań}
 % listy rozszerzone są do przepisania

% Route II
\chapter{Lab 71: IGP -- \textit{Interior Gateway Protocol}s, tunele GRE}
\section{Polecenia ogólne}
\cmdc{config}{ip routing}{Włączenie routingu IP}
\cmdc{config}{ip classless}{Włączenie routingu bezklasowego}
\cmdc{config}{ip route \textit{adres\_sieci} \textit{odwrócona\_maska} [\textit{dystans\_administracyjny}]}{Dodanie statycznego wpisu do tablicy routingu (niesprecyzowany dystans = 1)}
\cmdc{config-router}{no auto-summary}{Wyłączenie automatycznego uogólniania tras (supernettingu)}

\section{RIP -- \textit{Routing Information Protocol}}
Trasy dodawane przez RIP mają maksymalną metrykę równą 120.
\cmdc{config}{router rip}{Wejście w tryb konfiguracji RIP}

\subsection{RIPv1}
\cmdc{config-router}{version 1}{Włączenie wersji 1 protokołu (domyślne)}
\cmdc{config-router}{network \textit{adres\_sieci}}{Włączenie sieci do procesu routingu (routing klasowy)}

\subsection{RIPv2}
\cmdc{config-router}{version 2}{Włączenie wersji 2 protokołu}
\cmdc{config-router}{network \textit{adres\_sieci} \textit{odwrócona\_maska}}{Włączenie sieci do procesu routingu (routing bezklasowy)}
\cmdc{config-router}{no auto-summary}{Wyłączenie automatycznego uogólniania tras (supernettingu)}

\subsubsection{Selektywny wybór wersji protokołu}
\cmdc{config-if}{ip rip send version \{1, 2\}}{Wybór konkretnej wersji protokołu rozgłaszanej przez interfejs; dostępne wyłącznie dla RIPv2}
\cmdc{config-if}{ip rip receive version \{1, 2\}}{Wybór konkretnej wersji protokołu rozgłaszanej przez interfejs; dostępne wyłącznie dla RIPv2}

\subsubsection{Autentykacja}
\cmdc{config}{key chain \textit{nazwa\_keychain}}{Tworzenie nowego \texttt{keychain} (\textit{pęku kluczy})}
\cmdc{config-keychain}{key \textit{numer\_klucza}}{Dodawanie nowego klucza do \texttt{keychain}}
\cmdc{config-keychain-key}{key-string \textit{klucz}}{Ustawienie wartości klucza}
\cmdc{config-router}{ip authentication key-chain \textit{nazwa\_łańcucha\_kluczy}}{Wybór \texttt{keychain} używanego podczas autentykacji}
\cmdc{config-router}{ip authentication mode \{md5, sha1\}}{Wybór funkcji skrótu używanej podczas autentykacji}

\subsection{Diagnostyka}
\cmde{debug ip rip}{Włączenie debugowania RIP}

\section{OSPF -- \textit{Open Shortest Path First}}
Domyślne interwały \texttt{hello-time}: 10 sekund dla sieci typu broadcast oraz point-to-point, 30 - non-broadcast oraz point-to-multipoint.
\texttt{dead-time} jest 4 razy dłuższy niż \texttt{hello-time}.
\cmdc{config}{router ospf \textit{numer\_procesu}}{Wejście w tryb konfiguracji OSPF}
\cmdc{config-router}{network \textit{adres\_sieci} \textit{odwrócona\_maska} area \textit{numer\_obszaru}}{Włączenie sieci do procesu routingu}
\cmdc{config-router}{neighbor \textit{adres\_IP\_sąsiada}}{Rejestrowanie sąsiada OSPF (konieczne w przypadku sieci \textit{Non-Broadcast Multiple Access})}
\cmdc{config-router}{log-adjacency-changes}{Włącza przesyłanie komunikatów do \texttt{syslog}u podczas zmiany stanu sąsiada}

\subsection{Ustawienia pojedynczego interfejsu}
\cmdc{config-if}{ip ospf priority \textit{priorytet}}{Ustawienie priorytetu przy wyborze roli (DR, BDR) dla sieci typu broadcast (domyślnie 1) -- wyższa wartość wygrywa}
\cmdc{config-if}{ip ospf hello-interval \textit{interwał}}{Ustawienie interwału czasowego (w sekundach) dla pakietów \texttt{hello}; powoduje \textit{implicite} zmianę \texttt{dead-time}}
\cmdc{config-if}{ip ospf cost \textit{koszt}}{Zmiana kosztu naliczanego dla łącza \textit{wchodzącego do routera}}

\subsection{\textit{Virtual Link}}
Konfigurujemy, gdy router nie ma bezpośredniego dostępu do szkieletu -- tranzyt jest prowadzony przez inny obszar. Polecenia wpisujemy \textit{wyłącznie} na routerze będącym w szkielecie oraz obszarze docelowym. W poleceniu wpisujemy \texttt{area} znajdujące pomiędzy obszarami; identyfikatorem OSPF (to nie jest adres IP!) jest Router-ID \textit{przeciwległego} końca.
\cmdc{config-router}{area \textit{numer\_obszaru\_pośredniego} virtual-link \textit{przciwległy\_Router-ID}}{Konfiguracja \textit{Virtual Link}}

\subsubsection{Diagnostyka}
\cmde{show ip ospf virtual-links}{Wyświetlenie informacji o skonfigurowanych \textit{Virtual Link}s}

\subsection{Diagnostyka}
\cmde{show ip ospf interface \textit{numer\_interfejsu}}{Wyświetlenie informacji OSPF dotyczących wybranego interfejsu}
\cmde{show ip ospf neighbor}{Wyświetlenie bazy sąsiadów OSPF}
\cmde{show ip ospf border-routers}{Wyświetlenie routerów ABR, ASBR}

\subsubsection{Debugowanie}
\cmde{debug ip ospf adjacency}{Debugowanie zmian adjacencji (sąsiadów)}
\cmde{debug ip ospf events}{Debugowanie wydarzeń}

\section{Redystrybucja tras}
W trybie \texttt{(config-router)} polecenie \texttt{redistribute} oznacza redystrybucję za pomocą protokołu \textit{aktualnie konfigurowanego} informacji pochodzących z protokołu \textit{jawnie} zadanego.

\subsection{Redystrybucja za pomocą RIP}
\textbf{Podanie metryki jest konieczne}, ponieważ RIP wspiera \textit{maksymalnie} 15 przeskoków (jest to maksymalna wartość metryki). RIP nie flaguje tras pochodzących z obcego protokołu.
\cmdc{config-router}{redistribute ospf \textit{numer\_procesu\_OSPF} metric \textit{metryka}}{Redystrybucja tras pochodzących z OSPF przez RIP}

\subsection{Redystrybucja za pomocą OSPF}
Dodanie do konfiguracji parametru \texttt{subnets} powoduje, że redystrybuowane są sieci nieklasowe (VLSM).
\cmdc{config-router}{redistribute rip metric \textit{metryka} subnets}{Redystrybucja tras pochodzących z RIP przez OSPF}

\section{Tunele GRE -- \textit{Generic Routing Encapsulation}}
Konfigurację przeprowadzamy na obu końcach, zamieniając adresy celu i źródła.
Tunel jest warstwy trzeciej, tworzy zatem pełnoprawną sieć IP -- konieczne jest ustalenie wewnętrznej adresacji i skonfigurowanie interfejsów (np. za pomocą \texttt{ip address ...}).
\cmdc{config}{interface tunnel \textit{numer\_interfejsu}}{Tworzenie nowego interfejsu tunelu}
\cmdc{config-if}{tunnel source \textit{interfejs\_źródłowy}}{Określenie interfejsu, którym będą wychodziły enkapsulowane (tunelowane) pakiety GRE}
\cmdc{config-if}{tunnel destination \textit{adres\_IP\_końca}}{Określenie adresu IP odległego końca tunelu GRE}

\subsection{Zmiana MTU i TCP MSS clamping}
Tunelowanie powoduje nadanie dodatkowego narzutu (nagłówków GRE), efektywnie zmniejszając MTU oraz MSS.
\cmdc{config-if}{ip mtu \textit{wartość\_MTU}}{Zmiana MTU łącza}
\cmdc{config-if}{ip tcp adjust-mss \textit{wartość\_MSS}}{Zmiana MSS łącza (MSS clamping)}

\section{Diagnostyka}
\cmde{show ip route}{Wyświetlenie tablicy routingu}
\cmde{show ip protocols}{Wyświetla informacje uruchomionych protokołów routingu}
\cmde{debug ip routing}{Włączenie debugowania routingu}
\cmde{clear ip route *}{Czyszczenie tras routingu (kasuje statyczne wpisy); tablica jest populowana ponownie przez procesy routingu}
 % rozdzielić tunelowanie osobno

% Route III
\chapter{Lab 80: \texttt{route-map}s, Policy Based Routing, redundancja (HSRP, VRRP), IP SLA}
\section{Routing statyczny}
\cmdc{config}{ip route 0.0.0.0 0.0.0.0 \textit{adres\_IP\_następnego\_skoku}}{Dodawanie bramy domyślnej (ang. \textit{gateway of last resort})}
\cmdc{config}{ip route \textit{adres\_sieci} \textit{maska\_sieci} \textit{interfejs\_point-to-point} [\textit{dystans\_administracyjny}]}{Dodawanie trasy routingu przechodzącej przez interfejs point-to-point}
\cmde{clear ip route *}{Kasowanie całej tablicy reguł routowania TODO: sprawdzić, co ze statycznymi wpisami}
\cmdc{config-routers}{redistribute static}{Redystrybucja statycznych tras routingu za pomocą dynamicznego protokołu rutowania}

\section{\texttt{route-map}s i Policy Based Routing}
Route mapy wewnątrz korzystają z list ACL do klasyfikacji (\texttt{match}owania) ruchu, na którym mają operować. Numer sekwencji ma znaczenie podobne jak przy listach ACL -- rozpatrywane są po kolei (od najmniejszego, do największego numeru sekwencji); spełnienie reguły kończy dalszą klasyfikację. TODO: sprawdzić to

\subsection{Konfiguracja \texttt{route-map}y}
\cmdc{config}{route-map \textit{nazwa\_mapy} \textit{numer\_sekwencji}}{Tworzenie nowej \texttt{route-map}y}
\cmdc{config-route-map}{match ip address \textit{numer\_listy\_ACL}}{Ustalenie warunków dopasowania reguły}
\cmdc{config-route-map}{set interface \textit{nazwa\_interfejsu}}{Ustalenie interfejsu wyjściowego pakietu}
\cmdc{config-route-map}{set ip next-hop \textit{adres\_IP\_następnego\_routera}}{Ustalenie routera, do którego zostanie przekazany pakiet}

\subsection{Przypisanie \texttt{route-map}y do interfejsu}
\cmdc{config-if}{no ip route-cache}{Wyłączenie cache routingu}
\cmdc{config-if}{ip policy route-map \textit{nazwa\_mapy}}{Przypisanie \texttt{route-map}y do interfejsu}

\subsection{Diagnostyka}
\cmde{show ip policy}{Sprawdzenie stanu \textit{Policy Based Routing}u}
\cmde{debug ip policy}{Włączenie debugowania \textit{Policy Based Routing}u}
\cmde{show route-maps}{Wyświetlenie \texttt{route-map}}
\cmde{show access-lists}{Wyświetlenie list ACL}

\section{Rendundancja routerów}
Routery pracują w dwóch trybach: \textit{active} bądź \textit{standby}.
Oba protokoły wspierają autentykację. TODO: dodać link do appendix

\subsection{Cisco HSRP -- \textit{Hot Standby Routing Protocol}}
\cmdc{config-if}{standby \textit{numer\_grupy\_standby} ip \textit{adres\_IP\_standby}}{Ustalenie adresu IP standby (współdzelonego)}
\cmdc{config-if}{standby \textit{numer\_grupy\_standby} priority \textit{priorytet\_routera}}{Ustalenie priorytetu routera (większa wartość ma pierwszeństwo)}
\cmdc{config-if}{standby \textit{numer\_grupy\_standby} preempt}{Włączenie \textit{wywłaszczania} (po powrocie routera o wyższym priorytecie powraca on do trybu \textit{active})}

\subsubsection{Diagnostyka}
\cmde{show standby}{Wyświetlenie informacji o stanie HSRP}
\cmde{show mac-address-table int \textit{nazwa\_interfejsu}}{Na przełączniku podłączonym do obu routerów: sprawdzenie rozgłoszonych adresów MAC routerów}

\subsection{VRRP -- \textit{Virtual Routing Redundancy Protocol}}
\cmdc{config-if}{vrrp \textit{numer\_grupy\_VRRP} ip \textit{adres\_IP\_standby}}{Ustalenie adresu IP standby (współdzelonego)}
\cmdc{config-if}{vrrp \textit{numer\_grupy\_VRRP} priority \textit{priorytet\_routera}}{Ustalenie priorytetu routera (większa wartość ma pierwszeństwo)}
\cmdc{config-if}{vrrp \textit{numer\_grupy\_vrrp} preempt}{Włączenie \textit{wywłaszczania} (po powrocie routera o wyższym priorytecie powraca on do trybu \textit{active})}

\subsubsection{Zmiana priorytetu w skutek wydarzeń}
\cmdc{config}{track \textit{numer\_procesu\_śledzenia} interface \textit{nazwa\_interfejsu} line-protocol}{Zdefiniowanie procesu śledzenia zmiany stanu protokołu łącza}
\cmdc{config-if}{vrrp \textit{numer\_grupy\_VRRP} track \textit{numer\_procesu\_śledzenia} decrement \textit{różnica\_priorytetu}}{Zdefiniowanie zmniejszenie priorytetu w skutek zajścia zdarzenia}

\subsubsection{Diagnostyka}
\cmde{show vrrp}{Wyświetlenie informacji o stanie VRRP}

\subsection{Diagnostyka}
\cmde{show mac-address-table int \textit{nazwa\_interfejsu}}{Na przełączniku podłączonym do obu routerów: sprawdzenie rozgłoszonych adresów MAC routerów}

\section{IP SLA -- \textit{Service Level Agreement}}
\subsection{Cisco IOS 12.3}
\cmdc{config}{ip sla monitor \textit{numer\_opcji\_monitorowania}}{Definiowanie operacji monitorowania}
\cmdc{config-sla-monitor}{type echo protocol ip icmpEcho \textit{adres\_IP} source-interface \\ \textit{nazwa\_interfejsu\_źródłowego}}{Definiowanie monitorowania hosta poprzez wysyłanie pakietów \textit{ICMP Echo Request} (\texttt{ping})}
\cmdc{config-ip-sla}{timeout \textit{czas\_timeout}}{Zdefiniowanie czasu, po jakim host zostanie uznany jako niedostępny braku odpowiedzi}
\cmdc{config-ip-sla}{treshold 2}{TODO: sprawdzić}
\cmdc{config-ip-sla}{frequency 3}{TODO: sprawdzić}
\cmdc{config}{ip sla monitor schedule \textit{numer\_opcji\_monitorowania} life forever start-time now}{Uruchomienie monitorowania}

\subsection{Od Cisco IOS 12.4}
\cmdc{config}{ip sla \textit{numer\_opcji\_monitorowania}}{Definiowanie operacji monitorowania}
\cmdc{config-sla-monitor}{icmp-echo \textit{adres\_IP} source-interface \textit{nazwa\_interfejsu\_źródłowego}}{Definiowanie monitorowania hosta poprzez wysyłanie pakietów \textit{ICMP Echo Request} (\texttt{ping})}
\cmdc{config-ip-sla}{timeout \textit{czas\_timeout}}{Zdefiniowanie czasu, po jakim host zostanie uznany jako niedostępny braku odpowiedzi}
\cmdc{config-ip-sla}{treshold 2}{TODO: sprawdzić}
\cmdc{config-ip-sla}{frequency 3}{TODO: sprawdzić}
\cmdc{config}{ip sla schedule \textit{numer\_opcji\_monitorowania} life forever start-time now }{Uruchomienie monitorowania}

\subsection{Use case: konfiguracja bramy w zależności od dostępności hosta}
Konfigurowany jest tzw. RTR -- \textit{Response Time Reporter}. Oprócz trasy zdefiniowanej poniżej powinna istnieć inna, o \textbf{wyższej} metryce -- gdy host jest dostępny, brana jest trasa z parametrem \texttt{track} (o niższej metryce -- preferowana); w przeciwnym wypadku: ta z wyższą metryką.
\cmdc{config}{track \textit{numer\_obiektu\_śledzenia} rtr \textit{numer\_opcji\_monitorowania} reachability}{Konfiguracja reakcji na zdarzenie}
\cmdc{config TODO: sprawdzić}{ip route 0.0.0.0 0.0.0.0 \textit{adres\_IP\_bramy} track \textit{metryka\_gdy\_dostępny}}{Dodanie domyślnej trasy routingu gdy host jest dostępny}
\chapter{IP multicast}
\section{IGMP Snooping w przełącznikach Ethernet}
\cmdc{config}{ip igmp snooping}{Włączenie funkcji IGMP Snooping}

\section{PIM -- \textit{Protocol Independent Multicast}}
\cmdc{config}{ip multicast-routing}{Włączenie routingu multicast}
\cmdc{config-if}{ip igmp join-group \textit{adres\_multicast}}{Zdefiniowanie fikcyjnego źródła IP multicast}

\section{\texttt{dense-mode} (push mode)}
\cmdc{config-if}{ip pim dense-mode}{Włączenie trybu dense}

\subsection{Diagnostyka}
\cmde{show ip mroute}{Wyświetlanie tablicy routingu multicast}
\cmde{show ip igmp groups}{Wyświetlanie zarejestrowanych grup IGMP}
\cmde{show ip igmp grou membership}{Wyświetlenie tablicy grup IGMP}
\cmde{clear ip mroute *}{Usunięcie wpisów z tablicy routingu muticast}
\cmde{clear ip igmp groups}{Usunięcie wpisów z tablicy grup IGMP}

\section{\texttt{sparse-mode} (pull mode)}
Jeden z routerów pełni rolę \textit{Rendezvous Point} -- przechowuje informacje nt. lokalizacji źródeł multicast.
\cmdc{config}{ip pim rp-address \textit{adres\_IP\_RP}}{Wskazanie, pod jakim adresem znajduje się \textit{Rendezvous Point}}

\subsection{Diagnostyka}
\cmde{show ip pim rp}{TODO: sprawdzić}
\cmde{show ip pim rp mapping}{TODO: sprawdzić}
\cmde{show ip pim neighbor}{TODO: sprawdzić}

% NAT, IPv6
\chapter{IPv6}
\section{Konfiguracja interfejsów}
\cmdc{config-if}{ipv6 address \textit{adres\_IPv6}/\textit{maska\_podsieci}}{Konfiguracja adresu IPv6}
\cmdc{config-if}{ipv6 address \textit{adres\_IPv6\_sieci}/\textit{maska\_podsieci} eui-64}{Konfiguracja adresu IPv6 (EUI-64)}
\cmdc{config}{ipv6 route \textit{adres\_sieci\_IPv6} \textit{adres\_IPv6\_następnego\_skoku}}{arg3}

\section{Routing IPv6}
\cmdc{config}{ipv6 unicast-routing}{Włączenie routingu unicastowego IPv6}

\subsection{RIPng}
Wyjątkowo do procesu routowania nie dołączamy sieci (tj. za pomocą polecenia \texttt{network}), tylko pojedyncze \textit{interfejsy}.
\cmdc{config}{ipv6 router rip \textit{nazwa\_procesu}}{Włączenie RIPng}
\cmdc{config-if}{ipv6 rip \textit{nazwa\_procesu} enable}{Aktywacja RIPng na wybranym interfejsie}

\subsubsection{Diagnostyka}
\cmde{show ipv6 rip}{Wyświetlanie informacji o stanie RIPng}

\subsection{OSPFv3}
\cmdc{config}{ipv6 router ospf \textit{numer\_procesu}}{Włączenie OSPFv3}
\cmdc{config-router}{router-id \textit{aaa.bbb.ccc.ddd}}{Konfiguracja OSPF \texttt{router-id} TODO: sprawdzić, czy może być samym numerem}
\cmdc{config-if}{ipv6 ospf \textit{numer\_procesu} area \textit{aaa.bbb.ccc.ddd}}{Aktywacja OSPFv3 na wybranym interfejsie TODO: sprawdzić, czy może być po prostu liczbą}

\subsubsection{Diagnostyka}
\cmde{show ipv6 protocols}{TODO: sprawdzić}
\cmde{show ipv6 ospf neighbor}{Wyświetlenie informacji o sąsiadach OSPFv3}
\cmde{show ipv6 ospf interface}{Wyświetlenie informacji o interfejsach pracujących w ramach OSPFv3}
\cmde{show ipv6 ospf database}{Wyświetlenie bazy wiedzy OSPFv3}
\cmde{show ipv6 ospf border-routers}{Wyświetlenie bazy tras do znanych routerów ABR}
\cmde{debug ipv6 ospf}{Włączenie debugowania OSPFv3}
\cmde{debug ipv6 ospf packet}{Debugowanie ruchu OSPFv3}
\cmde{debug ipv6 ospf hello}{Debugowanie pakietów OSPFv3 Hello}

\subsection{EIGRP}
\cmdc{config}{ipv6 router eigrp \textit{numer\_AS}}{Włączenie EIGRP}
\cmdc{config-if}{ipv6 eigrp \textit{numer\_AS}}{Aktywacja EIGRP na wybranym interfejsie TODO: sprawdzić, czy faktycznie podaje się AS}

\section{Diagnostyka}
\cmde{show ipv6 interface brief}{Wyświetlenie podsumowania konfiguracji interfejsów (IPv6)}
\cmde{debug ipv6 icmp}{Włączenie debugowania pakietów ICMP IPv6}
\cmde{traceroute ipv6 \textit{adres\_hosta\_IPv6}}{\texttt{traceroute} w wersji IPv6}
\cmde{show ipv6 route}{TODO: sprawdzić, na czym polega debugowanie tras}

\section{Tunelowanie IPv6IP}
TODO: merge z ,,tunele GRE'', DRY
Tunel jest zwykłym interfejsem warstwy trzeciej -- oprócz konfiguracji podanej poniżej konieczne jest ustalenie wewnętrznej adresacji i skonfigurowanie interfejsu tunelu (tj. nadanie adresu za pomocą \texttt{ipv6 address ...}).
Konfiguracja identyczna jak w przypadku standardowego tunelu GRE, dodajemy jedynie następujące polecenia:
\cmdc{config-if}{tunnel mode ipv6ip}{Ustawienie trybu pracy tunelu na IPv6IP}
\chapter{Lab 73: NAT -- \textit{Network Address Translation}}

\section{NAT \texttt{overload}}
Ruch kontrolowany jest poprzez listy ACL; lista musi zezwalać na ruch wychodzący z sieci wewnętrznej.

\cmdc{config}{access-list \textit{numer\_listy} permit \textit{adres\_sieci} \textit{odwrócona\_maska}}{Dodanie przykładowej listy ACL, której celem jest określenie hostów znajdujących się po stronie \textit{inside}}
\cmdc{config}{ip nat inside source list \textit{numer\_listy\_ACL} interface \textit{nazwa\_interfejsu\_zewnętrznego} overload}{Uruchomienie agregacji adresów z sieci inside pod adresem IP zadanego interfejsu}
\cmdc{config-if}{ip nat inside}{Określenie interfejsu do sieci wewnętrznej (inside)}
\cmdc{config-if}{ip nat outside}{Określenie interfejsu do sieci zewnętrznej (outside)}

\section{Translacje statyczne, DMZ}
Translacja statyczna jest dwukierunkowa. Host zdefiniowany adresem \textit{po} podmianie adresu źródłowego nazywany jest \textit{hostem wirtualnym} implementowanym przez NAT.
\cmdc{config}{ip nat inside source static \textit{adres\_źródłowy\_przed\_NAT} \textit{adres\_źródłowy\_po\_NAT}}{Podmiana adresu źródłowego}

\section{NAPT -- translacje statyczne z przekierowaniem portów TCP/UDP}
\cmdc{config}{ip nat inside source static \{tcp, udp\} \textit{adres\_IP\_hosta\_inside} \textit{port\_hosta\_inside} \\ \textit{adres\_zewnętrzny} \textit{port\_zewnętrzny}}{Zdefiniowanie reguły statycznej NAT i portów TCP}

\subsection{Diagnostyka}
\cmde{debug ip tcp transactions}{Debugowanie znaczących pakietów TCP: zmiany stanu, retransmisje, duplikaty}

\section{Diagnostyka}
\cmde{show ip nat translations}{Wyświetlenie aktualnych translacji NAT}
\cmde{show ip nat statistics}{Wyświetlenie statystyk NAT}
\cmde{show ip access-lists}{Wyświetlenie list ACL}

\chapter{Lab 118: SNMP -- \textit{Simple Network Management Protocol}}
SNMP pracuje na porcie 162 UDP.

\cmdc{config}{snmp-server community \textit{nazwa\_community} \{ro, rw, view\}}{Dodawanie nowego \texttt{community}}

\subsection{SNMP Traps -- pułapki SNMP}
\cmdc{config}{snmp-server host \textit{adres\_IP\_hosta} \{1, 2c, 3\} \textit{nazwa\_community} [\textit{rodzaj\_komunikatu}]}{Dodawanie hosta, do którego wysyłane będą pułapki (konkretnego typu -- np. config)}
\cmdc{config}{snmp-server enable traps \textit{rodzaj\_zdarzenia}]}{Włączenie wyłapywania konkretnych zdarzeń}

\subsection{SNMP wersja 3}
\cmdc{config}{snmp-server view \textit{nazwa\_widoku} \textit{gałąź\_MIB} included}{Tworzenie nowego widoku}
\cmdc{config}{snmp-server group \textit{nazwa\_grupy} v\textit{wersja} priv read \textit{widok\_odczytu} write \textit{widok\_zapisu} notify \textit{widok\_notyfikacji}}{Utworzenie grupy z określonymi widokami odczytu, zapisu i notyfikacji}
\cmdc{config}{snmp-server user \textit{nazwa\_użytkownika} \textit{nazwa\_grupy} v\textit{wersja} auth \{md5, sha\} \textit{hasło} priv des56 \textit{klucz}}{Tworzenie nowego użytkownika SNMP}

\subsubsection{Zarządzanie uprawnieniami}
\cmdc{config}{snmp-server group \textit{nazwa\_grupy} v3 auth \{read, write\} \textit{nazwa\_widoku} access \textit{numer\_listy\_ACL}}{Zarządzanie uprawnieniami poprzez listę ACL}

\subsection{Zdalne logowanie komunikatów systemowych}
\cmdc{config}{logging trap debugging}{Włączenie systemu logowania dla komunikatów \texttt{debug}}
\cmdc{config}{logging facility \{syslog, \textit{adres\_IP\_hosta}\}}{Wybór sposobu loggowania}
\cmdc{config}{logging file flash:\textit{nazwa\_pliku} debugging}{Logowanie do pliku}

\subsection{Diagnostyka}
\cmde{show snmp mib}{Wyświetlenie bazy MIB SNMP}
\cmde{show snmp host}{Wyświetlenie skonfigurowanych hostów SNMP}
\cmde{show snmp user}{Wyświetlenie użytkowników SNMP}
\cmde{show snmp view}{Wyświetlenie widoków SNMP}
\cmde{debug snmp packets}{Debugowanie pakietów SNMP}

% BGP
\chapter{Lab 77, 78: BGP -- \textit{Border Gateway Protocol}}
\cmdc{config}{router bgp \textit{numer\_AS}}{Wejście w tryb konfiguracji BGP}
\cmdc{config-router}{bgp router-id \textit{aaa.bbb.ccc.ddd}}{Konfiguracja BGP \texttt{router-id}}
\cmdc{config-router}{neighbor \textit{adres\_IP\_sąsiada} remote-as \textit{numer\_AS}}{Rejestrowanie sąsiada BGP}
\cmdc{config-router}{network \textit{adres\_sieci} [netmask \textit{maska\_podsieci}]}{Włączenie sieci do procesu routingu}
\cmdc{config-router}{no synchronization}{Wyłączenie synchronizacji z protokołami IGP -- wyłączenie oczekiwania na zestawienie ścieżki przez protokoły IGP; można wpisać, gdy nie jesteśmy multihomed}
\cmdc{config-router}{no auto-summary}{Wyłączenie automatycznego uogólniania tras (supernettingu)}

\section{Manipulowanie atrybutami tras eBGP}
Atrybuty modyfikujemy poprzez \texttt{route-map}s.
\cmdc{config}{route-map \textit{nazwa\_mapy}}{Tworzenie nowej \texttt{route-map}}
\cmdc{config-route-map}{set metric \textit{metryka}}{Modyfikacja wartości metryki}
\cmdc{config-route-map}{set origin incomplete}{Zadeklarowanie źródła jako spoza protokołu routingu}
\cmdc{config-route-map}{set local-preference \textit{wartość\_parametru}}{Modyfikacja \texttt{local-preference} (nie eksportuje się!)}
\cmdc{config-route-map}{set weight \textit{waga}}{Modyfikacja \texttt{weight} (nie eksportuje się!)}
\cmdc{config-route-map}{set as-path prepend \textit{numer\_AS1} \textit{numer\_AS2} ...}{Dodanie dodatkowych AS do trasy (z przodu ścieżki)}

\section{Filtrowanie tras eBGP}
\cmdc{config}{ip as-path access-list \textit{numer\_listy\_ACL} deny \_\textit{numer\_AS}\_}{Filtrowanie tras przechodzących przez zadany AS}
\cmdc{config}{ip as-path access-list \textit{numer\_listy\_ACL} permit \$\^}{Wydanie zezwolenie na pozostałe (dowolne) trasy}
\cmdc{config-router}{neighbor \textit{adres\_IP\_sąsiada} filter-list \textit{numer\_listy\_ACL} \{in, out\}}{Aplikacja filtra tras na zadanym sąsiedzie}

\subsection{Aplikacja \texttt{route-map}y}
\texttt{route-map}ę możemy zastosować do naszej sieci, sąsiada dla danych wysłanych i (osobno) odbieranych.
\cmdc{config-router}{network \textit{adres\_IP\_sieci} route-map \textit{nazwa\_mapy}}{Modyfikacja atrybutów tras wychodzących z tej sieci}
\cmdc{config-router}{neighbor \textit{adres\_IP\_sąsiada} route-map \textit{nazwa\_mapy} \{in, out\}}{Modyfikacja atrybutów przychodzących/wychodzących z/do sąsiada}

\section{Techniki skalowania iBGP}
\subsection{\textit{Route Reflector}}
Po wprowadzeniu sąsiada deklarujemy, że jest klientem RR.
\cmdc{config-router}{neighbor \textit{adres\_IP\_sąsiada} router-reflector-client}{Zadeklarowanie, że sąsiad jest klientem RR}
\cmdc{config-router}{bgp cluster \textit{identyfikator\_klastra}}{Definiowanie klastra RR (w przypadku, że istnieje więcej, niż jeden RR)}

\subsection{\textit{AS Confederations}}
\cmdc{config-router}{bgp confederation identifier \textit{numer\_AS\_zewnętrzny}}{Zgłoszenie routera z konfederacji do innych AS}
\cmdc{config-router}{bgp confederation peers \textit{numer\_innego\_AS\_w\_konfederacji}}{Zgłoszenie innych AS będących w konfederacji}

% TODO: dokończyć zadanie D

\section{Diagnostyka}
\cmde{show ip bgp}{Wyświetla podsumowanie stanu BGP (listę wpisów tablicy routowania BGP)}
\cmde{show ip bgp neighbors}{Wyświetlenie sąsiadów BGP}

\cmde{show ip protocols}{Wyświetla informacje uruchomionych protokołów routingu}
\cmde{show ip bgp update-group}{Wyświetlanie \texttt{update-group}s -- grup sąsiadów, którzy otrzymują te same aktualizacje}
\cmde{clear ip bgp *}{Wyczyszczenie sesji BGP}
\cmde{debug ip bgp}{Włączenie debugowania sesji BGP} % brak zadania D

% CORE+EDGE II
\chapter{Lab 56: HP ProCurve}

\section{Konfiguracja}

\subsection{Ogólne}
\cmdc{config}{telnet-server}{Włączenie serwera telnet}
\cmdc{config}{no telnet}{Wyłączenie serwera telnet}

\cmdc{config}{web-management}{Włączenie serwera WWW}

\cmdc{config}{ip ssh key-size \textit{rozmiar\_klucza} }{Ustawienie rozmiaru klucza SSH}
\cmdc{config}{crypto key generate}{Generowanie klucza SSH (async)}
\cmdc{config}{ip ssh port \textit{numer\_portu} }{Ustawienie portu na którym ma działać serwer SSH}
\cmdc{config}{ip ssh}{Włączenie serwera SSH}

\cmdc{config}{interface \textit{numer}}{Przejście do konfiguracji portu}
\cmdc{config}{ip route \textit{netowork} \textit{mask} reject}{Zdefiniowanie reguły null}

\subsection{VLAN}
W przypadku przełączników HP w przeciwieństwie do tych Cisco przypisujemy porty do vlan'ów a nie vlan'y do portów

\cmdc{vlan-\textit{numer}}{tagged \textit{numer\_lub\_zakres\_portów}}{Przypisanie portu do Vlana}

\section{Diagnostyka}

\subsection{Ogólne}
\cmde{show telnet}{Wyświetlenie informacji o stanie serwera telnet}
\cmde{show ip ssh}{Wyświetlenie informacji o stanie serwera SSH}

\subsection{VLAN}
\cmde{show vlan}{Wyświetlenie informacji o stanie bazy vlan}
\cmde{show vlan \textit{numer}}{Wyświetlenie informacji konkretnym vlan'ie}
\chapter{ESW -- \textit{Embeded switch}}

ESW jest to moduł switcha zamontowany w routerze Cisco pozwala to na połączenie funkcjonalności switcha i routera w jednym urządzeniu. Zależnie od wybranego trybu pracy wybranego portu możemy konfigurować go jak w przełączniku lub jak w routerze.

\cmdc{config-if}{no switchport}{Zmiana trybu pracy portu na routowanie}
\cmdc{config-if}{switchport}{Zmiana tryby pracy portu na przełączanie}

%\begin{appendices}
%\include{appendix_crypto}
%\end{appendices}

\end{document}
