\chapter{Lab 73: NAT -- \textit{Network Address Translation}}

\section{NAT \texttt{overload}}
Ruch kontrolowany jest poprzez listy ACL; lista musi zezwalać na ruch wychodzący z sieci wewnętrznej.

\cmdc{config}{access-list \textit{numer\_listy} permit \textit{adres\_sieci} \textit{odwrócona\_maska}}{Dodanie przykładowej listy ACL, której celem jest określenie hostów znajdujących się po stronie \textit{inside}}
\cmdc{config}{ip nat inside source list \textit{numer\_listy\_ACL} interface \textit{nazwa\_interfejsu\_zewnętrznego} overload}{Uruchomienie agregacji adresów z sieci inside pod adresem IP zadanego interfejsu}
\cmdc{config-if}{ip nat inside}{Określenie interfejsu do sieci wewnętrznej (inside)}
\cmdc{config-if}{ip nat outside}{Określenie interfejsu do sieci zewnętrznej (outside)}

\section{Translacje statyczne, DMZ}
Translacja statyczna jest dwukierunkowa. Host zdefiniowany adresem \textit{po} podmianie adresu źródłowego nazywany jest \textit{hostem wirtualnym} implementowanym przez NAT.
\cmdc{config}{ip nat inside source static \textit{adres\_źródłowy\_przed\_NAT} \textit{adres\_źródłowy\_po\_NAT}}{Podmiana adresu źródłowego}

\section{NAPT -- translacje statyczne z przekierowaniem portów TCP/UDP}
\cmdc{config}{ip nat inside source static \{tcp, udp\} \textit{adres\_IP\_hosta\_inside} \textit{port\_hosta\_inside} \\ \textit{adres\_zewnętrzny} \textit{port\_zewnętrzny}}{Zdefiniowanie reguły statycznej NAT i portów TCP}

\subsection{Diagnostyka}
\cmde{debug ip tcp transactions}{Debugowanie znaczących pakietów TCP: zmiany stanu, retransmisje, duplikaty}

\section{Diagnostyka}
\cmde{show ip nat translations}{Wyświetlenie aktualnych translacji NAT}
\cmde{show ip nat statistics}{Wyświetlenie statystyk NAT}
\cmde{show ip access-lists}{Wyświetlenie list ACL}
