\chapter{Lab 10: IP multicast}
\section{IGMP Snooping w przełącznikach Ethernet}
\cmdc{config}{ip igmp snooping}{Włączenie funkcji IGMP Snooping}

\section{PIM -- \textit{Protocol Independent Multicast}}
\cmdc{config}{ip multicast-routing}{Włączenie routingu multicast}
\cmdc{config-if}{ip igmp join-group \textit{adres\_multicast}}{Zdefiniowanie fikcyjnego źródła IP multicast}

\section{\texttt{dense-mode} (push mode)}
\cmdc{config-if}{ip pim dense-mode}{Włączenie trybu dense}

\subsection{Diagnostyka}
\cmde{show ip mroute}{Wyświetlanie tablicy routingu multicast}
\cmde{show ip igmp groups}{Wyświetlanie zarejestrowanych grup IGMP}
\cmde{show ip igmp grou membership}{Wyświetlenie tablicy grup IGMP}
\cmde{clear ip mroute *}{Usunięcie wpisów z tablicy routingu muticast}
\cmde{clear ip igmp groups}{Usunięcie wpisów z tablicy grup IGMP}

\section{\texttt{sparse-mode} (pull mode)}
Jeden z routerów pełni rolę \textit{Rendezvous Point} -- przechowuje informacje nt. lokalizacji źródeł multicast.
\cmdc{config}{ip pim rp-address \textit{adres\_IP\_RP}}{Wskazanie, pod jakim adresem znajduje się \textit{Rendezvous Point}}

\subsection{Diagnostyka}
\cmde{show ip pim rp}{Wyświetla znane RP z powiązanymi trasami routingu multicast}
\cmde{show ip pim rp mapping}{Wyświetla powiązania grupy PIM (multicast) z RP}
\cmde{show ip pim neighbor}{Wyświetla znanych sąsiadów PIM (z którymi były wymieniane pakiety query/hello)}