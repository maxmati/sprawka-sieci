\chapter{Lab 76: IPv6}
\section{Konfiguracja interfejsów}
\cmdc{config-if}{ipv6 address \textit{adres\_IPv6}/\textit{maska\_podsieci}}{Konfiguracja adresu IPv6}
\cmdc{config-if}{ipv6 address \textit{adres\_IPv6\_sieci}/\textit{maska\_podsieci} eui-64}{Konfiguracja adresu IPv6 (EUI-64)}
\cmdc{config}{ipv6 route \textit{adres\_sieci\_IPv6} \textit{adres\_IPv6\_następnego\_skoku}}{arg3}

\section{Routing IPv6}
\cmdc{config}{ipv6 unicast-routing}{Włączenie routingu unicastowego IPv6}

\subsection{RIPng}
Wyjątkowo do procesu routowania nie dołączamy sieci (tj. za pomocą polecenia \texttt{network}), tylko pojedyncze \textit{interfejsy}.
\cmdc{config}{ipv6 router rip \textit{nazwa\_procesu}}{Włączenie RIPng}
\cmdc{config-if}{ipv6 rip \textit{nazwa\_procesu} enable}{Aktywacja RIPng na wybranym interfejsie}

\subsubsection{Diagnostyka}
\cmde{show ipv6 rip}{Wyświetlanie informacji o stanie RIPng}

\subsection{OSPFv3}
\cmdc{config}{ipv6 router ospf \textit{numer\_procesu}}{Włączenie OSPFv3}
\cmdc{config-router}{router-id \textit{aaa.bbb.ccc.ddd}}{Konfiguracja OSPF \texttt{router-id} (zapis jak adres IPv4, ponieważ jest to wartość 32-bitowa)}
\cmdc{config-if}{ipv6 ospf \textit{numer\_procesu} area \textit{numer\_obszaru}}{Aktywacja OSPFv3 na wybranym interfejsie}

\subsubsection{Diagnostyka}
\cmde{show ipv6 protocols}{Wyświetlenie stanu protokołów routingu IPv6}
\cmde{show ipv6 ospf neighbor}{Wyświetlenie informacji o sąsiadach OSPFv3}
\cmde{show ipv6 ospf interface}{Wyświetlenie informacji o interfejsach pracujących w ramach OSPFv3}
\cmde{show ipv6 ospf database}{Wyświetlenie bazy wiedzy OSPFv3}
\cmde{show ipv6 ospf border-routers}{Wyświetlenie bazy tras do znanych routerów ABR}
\cmde{debug ipv6 ospf}{Włączenie debugowania OSPFv3}
\cmde{debug ipv6 ospf packet}{Debugowanie ruchu OSPFv3}
\cmde{debug ipv6 ospf hello}{Debugowanie pakietów OSPFv3 Hello}

\subsection{EIGRP}
\cmdc{config}{ipv6 router eigrp \textit{numer\_AS}}{Włączenie EIGRP}
\cmdc{config-if}{ipv6 eigrp \textit{numer\_AS}}{Aktywacja EIGRP na wybranym interfejsie}

\section{Diagnostyka}
\cmde{show ipv6 interface brief}{Wyświetlenie podsumowania konfiguracji interfejsów (IPv6)}
\cmde{debug ipv6 icmp}{Włączenie debugowania pakietów ICMP IPv6}
\cmde{traceroute ipv6 \textit{adres\_hosta\_IPv6}}{\texttt{traceroute} w wersji IPv6}
\cmde{show ipv6 route}{Wyświetlenie tablicy routingu IPv6}

\section{Tunelowanie IPv6IP}
% TODO: merge z ,,tunele GRE'', DRY
Tunel jest zwykłym interfejsem warstwy trzeciej -- oprócz konfiguracji podanej poniżej konieczne jest ustalenie wewnętrznej adresacji i skonfigurowanie interfejsu tunelu (tj. nadanie adresu za pomocą \texttt{ipv6 address ...}).
Konfiguracja identyczna jak w przypadku standardowego tunelu GRE, dodajemy jedynie następujące polecenie:
\cmdc{config-if}{tunnel mode ipv6ip}{Ustawienie trybu pracy tunelu na IPv6IP}