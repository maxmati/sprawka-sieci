\chapter{Lab 80: \texttt{route-map}s, Policy Based Routing, redundancja (HSRP, VRRP), IP SLA}
\section{Routing statyczny}
\cmdc{config}{ip route 0.0.0.0 0.0.0.0 \textit{adres\_IP\_następnego\_skoku}}{Dodawanie bramy domyślnej (ang. \textit{gateway of last resort})}
\cmdc{config}{ip route \textit{adres\_sieci} \textit{maska\_sieci} \textit{interfejs\_point-to-point} [\textit{dystans\_administracyjny}]}{Dodawanie trasy routingu przechodzącej przez interfejs point-to-point}
\cmde{clear ip route *}{Czyszczenie tras routingu (kasuje statyczne wpisy); tablica jest populowana ponownie przez procesy routingu}
\cmdc{config-routers}{redistribute static}{Redystrybucja statycznych tras routingu za pomocą dynamicznego protokołu rutowania}

\section{\texttt{route-map}s i Policy Based Routing}
Route mapy wewnątrz korzystają z list ACL do klasyfikacji (\texttt{match}owania) ruchu, na którym mają operować. Numer sekwencji ma znaczenie podobne jak przy listach ACL -- rozpatrywane są po kolei (od najmniejszego, do największego numeru sekwencji); spełnienie reguły kończy dalszą klasyfikację. Jeżeli żadna reguła nie zostanie dopasowana, ruch jest odrzucany.

\subsection{Konfiguracja \texttt{route-map}y}
\cmdc{config}{route-map \textit{nazwa\_mapy} \textit{numer\_sekwencji}}{Tworzenie nowej \texttt{route-map}y}
\cmdc{config-route-map}{match ip address \textit{numer\_listy\_ACL}}{Ustalenie warunków dopasowania reguły}
\cmdc{config-route-map}{set interface \textit{nazwa\_interfejsu}}{Ustalenie interfejsu wyjściowego pakietu}
\cmdc{config-route-map}{set ip next-hop \textit{adres\_IP\_następnego\_routera}}{Ustalenie routera, do którego zostanie przekazany pakiet}

\subsection{Przypisanie \texttt{route-map}y do interfejsu}
\cmdc{config-if}{no ip route-cache}{Wyłączenie cache routingu}
\cmdc{config-if}{ip policy route-map \textit{nazwa\_mapy}}{Przypisanie \texttt{route-map}y do interfejsu}

\subsection{Diagnostyka}
\cmde{show ip policy}{Sprawdzenie stanu \textit{Policy Based Routing}u}
\cmde{debug ip policy}{Włączenie debugowania \textit{Policy Based Routing}u}
\cmde{show route-maps}{Wyświetlenie \texttt{route-map}}
\cmde{show access-lists}{Wyświetlenie list ACL}

\section{Rendundancja routerów}
Routery pracują w dwóch trybach: \textit{active} bądź \textit{standby}.
Oba protokoły wspierają autentykację.
% TODO: dodać link do appendix

\subsection{Cisco HSRP -- \textit{Hot Standby Routing Protocol}}
\cmdc{config-if}{standby \textit{numer\_grupy\_standby} ip \textit{adres\_IP\_standby}}{Ustalenie adresu IP standby (współdzelonego)}
\cmdc{config-if}{standby \textit{numer\_grupy\_standby} priority \textit{priorytet\_routera}}{Ustalenie priorytetu routera (większa wartość ma pierwszeństwo)}
\cmdc{config-if}{standby \textit{numer\_grupy\_standby} preempt}{Włączenie \textit{wywłaszczania} (po powrocie routera o wyższym priorytecie powraca on do trybu \textit{active})}

\subsubsection{Diagnostyka}
\cmde{show standby}{Wyświetlenie informacji o stanie HSRP}
\cmde{show mac-address-table int \textit{nazwa\_interfejsu}}{Na przełączniku podłączonym do obu routerów: sprawdzenie rozgłoszonych adresów MAC routerów}

\subsection{VRRP -- \textit{Virtual Routing Redundancy Protocol}}
\cmdc{config-if}{vrrp \textit{numer\_grupy\_VRRP} ip \textit{adres\_IP\_standby}}{Ustalenie adresu IP standby (współdzelonego)}
\cmdc{config-if}{vrrp \textit{numer\_grupy\_VRRP} priority \textit{priorytet\_routera}}{Ustalenie priorytetu routera (większa wartość ma pierwszeństwo)}
\cmdc{config-if}{vrrp \textit{numer\_grupy\_vrrp} preempt}{Włączenie \textit{wywłaszczania} (po powrocie routera o wyższym priorytecie powraca on do trybu \textit{active})}

\subsubsection{Zmiana priorytetu w skutek wydarzeń}
\cmdc{config}{track \textit{numer\_procesu\_śledzenia} interface \textit{nazwa\_interfejsu} line-protocol}{Zdefiniowanie procesu śledzenia zmiany stanu protokołu łącza}
\cmdc{config-if}{vrrp \textit{numer\_grupy\_VRRP} track \textit{numer\_procesu\_śledzenia} decrement \textit{różnica\_priorytetu}}{Zdefiniowanie zmniejszenie priorytetu w skutek zajścia zdarzenia}

\subsubsection{Diagnostyka}
\cmde{show vrrp}{Wyświetlenie informacji o stanie VRRP}

\subsection{Diagnostyka}
\cmde{show mac-address-table int \textit{nazwa\_interfejsu}}{Na przełączniku podłączonym do obu routerów: sprawdzenie rozgłoszonych adresów MAC routerów}

\section{IP SLA -- \textit{Service Level Agreement}}
\subsection{Cisco IOS 12.3}
\cmdc{config}{ip sla monitor \textit{numer\_opcji\_monitorowania}}{Definiowanie operacji monitorowania}
\cmdc{config-sla-monitor}{type echo protocol ip icmpEcho \textit{adres\_IP} source-interface \\ \textit{nazwa\_interfejsu\_źródłowego}}{Definiowanie monitorowania hosta poprzez wysyłanie pakietów \textit{ICMP Echo Request} (\texttt{ping})}
\cmdc{config-ip-sla}{timeout \textit{czas\_timeout}}{Maksymalny czas odpowiedzi (w sekundach); jeżeli nie otrzymamy odpowiedzi w zadanym czasie, host uznawany jest jako niedostepny}
\cmdc{config-ip-sla}{threshold \textit{threshold\_w\_ms}}{Górna granica wartości odpowiedzi}
\cmdc{config-ip-sla}{frequency \textit{częstotliwość\_w\_s}}{Częstotliwość wykonywania zapytania}
\cmdc{config}{ip sla monitor schedule \textit{numer\_opcji\_monitorowania} life forever start-time now}{Uruchomienie monitorowania}

\subsection{Od Cisco IOS 12.4}
\cmdc{config}{ip sla \textit{numer\_opcji\_monitorowania}}{Definiowanie operacji monitorowania}
\cmdc{config-sla-monitor}{icmp-echo \textit{adres\_IP} source-interface \textit{nazwa\_interfejsu\_źródłowego}}{Definiowanie monitorowania hosta poprzez wysyłanie pakietów \textit{ICMP Echo Request} (\texttt{ping})}
\cmdc{config-ip-sla}{timeout \textit{czas\_timeout}}{Maksymalny czas odpowiedzi (w sekundach); jeżeli nie otrzymamy odpowiedzi w zadanym czasie, host uznawany jest jako niedostepny}
\cmdc{config-ip-sla}{threshold \textit{threshold\_w\_ms}}{Górna granica wartości odpowiedzi}
\cmdc{config-ip-sla}{frequency \textit{częstotliwość\_w\_s}}{Częstotliwość wykonywania zapytania}
\cmdc{config}{ip sla schedule \textit{numer\_opcji\_monitorowania} life forever start-time now }{Uruchomienie monitorowania}

\subsection{Use case: konfiguracja bramy w zależności od dostępności hosta}
Konfigurowany jest tzw. RTR -- \textit{Response Time Reporter}. Oprócz trasy zdefiniowanej poniżej powinna istnieć inna, o \textbf{wyższej} metryce -- gdy host jest dostępny, brana jest trasa z parametrem \texttt{track} (o niższej metryce -- preferowana); w przeciwnym wypadku: ta z wyższą metryką.
\cmdc{config}{track \textit{numer\_obiektu\_śledzenia} rtr \textit{numer\_opcji\_monitorowania} reachability}{Konfiguracja reakcji na zdarzenie}
\cmdc{config}{ip route 0.0.0.0 0.0.0.0 \textit{adres\_IP\_bramy} track \textit{metryka\_gdy\_dostępny}}{Dodanie domyślnej trasy routingu gdy host jest dostępny}